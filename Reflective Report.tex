% Please do not change the document class
\documentclass{scrartcl}

% Please do not change these packages
\usepackage[hidelinks]{hyperref}
\usepackage[none]{hyphenat}
\usepackage{setspace}
\usepackage{graphicx}
\doublespace

% You may add additional packages here
\usepackage{amsmath}

% Please include a clear, concise, and descriptive title
\title{Reflective Report for Time Management, Critical Evaluation and Communication}

% Please do not change the subtitle
\subtitle{COMP120 - Reflective Report}

% Please put your student number in the author field
\author{1503048}

\begin{document}

\maketitle

\section{Introduction}

The three weakness I explore in this report are: time management, critical evaluation and communication. All of these skills are significant to me for professional practise and have been emphasised during my first semester.

\section{Time Management}

One of my weaknesses was time management. With most tasks, it is crucial to follow a strict and thoughtful plan which accounts for all aspects of the project. For me, the relevance of this issue was for not anticipating all aspects of each project.

I found that my allocated work time had been misplaced. This not only lead to a low quality of work but also low quantity.

Initially, I noticed that I did not feel confident in my abilities. At the time, I did not know how to overcome this issue which lead to frustration.

I now realise my time should have been more focused. For the future, I aim to make structured session objectives and expected completion times. If objectives are incomplete, then I must approach my lecturer or colleagues with my concerns. Next month, I shall examine my expectations of each week and whether my objectives were completed or guidance was sought. This focus would greatly contribute to my future career as a game developer.

\section{Critical Evaluation}

Another of my weaknesses was critical evaluation. The ability to critique my own work allows for issues to be discussed and improvements to be made.

Unlike my previous education, these tasks required my own objective evaluation of work quality. This lead to being overly critical of my work, in a way that wasn't constructive.

For me, the issue arose from low confidence. Subsequently, I felt inappropriately equipped to offer critical advice. In addition to this, I felt unfamiliar with marking others work with regards to coding practise.

I now realise that making reference to project objectives focuses evaluation. In addition, using objectives as a checklist proves each topic has been discussed. For me, in a career as a game developer, this skill would provide focused group and individual reflection. Next month, I shall examine my peer reviews as to whether I covered, and suggested improvements, for all points in my checklist.

\section{Communication}

A final weakness is communication, which is relevant for all projects. In addition to verbal presentations, clearly written documents and code comments are crucial for project work flow. In group work, I failed to engage in discussion which lead to poorly assembled presentations.

Documentation and clear comments would have greatly improved my learning practise. It also would have increased my confidence and knowledge of my own work.

For me, this aspect is crucial for future employment. A portfolio with code comprehension, through appropriate commenting, would improve my employability as a programmer.

After using Github, I feel confident using short description to identify issues or variables. Next month, I shall examine my Github submissions to ensure I followed appropriate naming conventions while adding comments during programming sessions.

\section{Conclusion}

For future learning, and professional practise, I aim to use SMART actions to ensure measureable and achievable standards of work.

%\bibliographystyle{ieeetr}
%\bibliography{PCG_export}

\end{document}